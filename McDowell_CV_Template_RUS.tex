%% The MIT License (MIT)
%%
%% Copyright (c) 2015 Daniil Belyakov
%%
%% Permission is hereby granted, free of charge, to any person obtaining a copy
%% of this software and associated documentation files (the "Software"), to deal
%% in the Software without restriction, including without limitation the rights
%% to use, copy, modify, merge, publish, distribute, sublicense, and/or sell
%% copies of the Software, and to permit persons to whom the Software is
%% furnished to do so, subject to the following conditions:
%%
%% The above copyright notice and this permission notice shall be included in all
%% copies or substantial portions of the Software.
%%
%% THE SOFTWARE IS PROVIDED "AS IS", WITHOUT WARRANTY OF ANY KIND, EXPRESS OR
%% IMPLIED, INCLUDING BUT NOT LIMITED TO THE WARRANTIES OF MERCHANTABILITY,
%% FITNESS FOR A PARTICULAR PURPOSE AND NONINFRINGEMENT. IN NO EVENT SHALL THE
%% AUTHORS OR COPYRIGHT HOLDERS BE LIABLE FOR ANY CLAIM, DAMAGES OR OTHER
%% LIABILITY, WHETHER IN AN ACTION OF CONTRACT, TORT OR OTHERWISE, ARISING FROM,
%% OUT OF OR IN CONNECTION WITH THE SOFTWARE OR THE USE OR OTHER DEALINGS IN THE
%% SOFTWARE.

% The font could be set to Windows-specific Calibri by using the 'calibri' option
\documentclass[]{mcdowellcv}

\makeatletter
\renewenvironment{cvsubsection}[2]{%
  \begin{adjustwidth}{\subsectionmargin}{\subsectionmargin}%
    {\bfseries #1}\hfill #2\par\vspace{0.5em}%
}{%
  \end{adjustwidth}%
  \vspace*{\aftersubsectionspace}%
}
\renewcommand{\makeheader}{%
  \begin{center}
    \printname\\[0.5em]%
    \printcontacts%
  \end{center}%
  \vspace*{\afterheaderspace}%
}
\makeatother


% For mathematical symbols
\usepackage{amsmath}
\usepackage[hidelinks]{hyperref}
\usepackage{setspace}

% Увеличиваем межстрочный интервал
\linespread{1.2}\selectfont
%\onehalfspacing

% Меняем интервал между абзацами (если нужно)
\setlength{\parskip}{0.75em}


% Set applicant's personal data for header
\name{\huge Ренат Юнисов}
% \contacts{+7(977)285-32-86 \linebreak ryunisov0@gmail.com \linebreak @rryunisov}
\contacts{\href{ryunisov0@gmail.com}{\underline{ryunisov0@gmail.com}} \hspace{6.5em} \quad \href{https://t.me/rryunisov}{\underline{@rryunisov}} \hspace{6.5em} \quad \href{https://www.linkedin.com/in/renat-yunisov-m02/}{\underline{LinkedIn}} \hspace{6.5em} \quad \href{https://github.com/Renarion}{\underline{GitHub}}}


\begin{document}

    % Print the header
    \makeheader
    
    % Print the content
    \begin{cvsection}{\Large Опыт работы}
        \begin{cvsubsection}{inDrive, Trust \& Safety, Продуктовый аналитик}{\textit{Сентябрь 2024 -- Настоящее время}}
            \begin{itemize}
                \item Создал систему графов связанных аккаунтов с агрессорами, это позволило разработать автоматические баны и сократить количество повторных инцидентов на \textbf{\textasciitilde ~15\%} в Латинской Америке;
                \item Обучил модель скоринга безопасности пассажиров на основании 50 параметром и получил Precision на уровне \textbf{30\%} на несбалансированном выборке. Это помогло запустить процесс создания хранилища параметров для реализации сложных моделей на проде;
                \item Разработал модель мониторинга отклонений во время поездки с использованием 5 параметров, что увеличило поток новых обращений об инцидентах на \textbf{3к в месяц (\textasciitilde ~5\%)} и помогло узнавать о них раньше;
                \item Запустил более 15 экспериментов и разработал библиотеку для A/B тестирования на Python, что позволило сократить время на подведение результатов примерно на \textbf{25\%} (\href{https://github.com/Renarion/expab}{GitHub}).
            \end{itemize}
        \end{cvsubsection}
        
        \begin{cvsubsection}{InvestEngine, Продуктовый аналитик}{\textit{Март 2024 -- Август 2024}}
            \begin{itemize}
                \item Выстроил систему клиентских событий с нуля и запустил интеграцию с Google BigQuery;
                \item Создал аналитическую документацию и разработал фреймворк с \textbf{более чем 10} продуктовыми метриками, который стал основой для KPI операционной команды.
            \end{itemize}
        \end{cvsubsection}
        
        \begin{cvsubsection}{Tinkoff (T-Bank), SME, Продуктовый аналитик}{\textit{Март 2023 -- Март 2024}}
            \begin{itemize}
                \item Провел исследование и запустил эксперимент по внедрению нового процесса обработки заявок на создание индивидуального предпринимательства, что помогло увеличить CR утилизации на \textbf{\textasciitilde ~11\%} в течение 7 дней и принесло доход более \textbf{1 млн. руб.};
                \item Выявил причину возросшего оттока по дебетовым (бизнес) картам и содействовал в разработке маркетинговых кампаний, которые сократили этот показатель примерно на \textbf{\textasciitilde 3\%};
            \end{itemize}
        \end{cvsubsection}
        
        \begin{cvsubsection}{Ozon, Экспресс доставка, Аналитик данных}{\textit{Август 2021 -- Март 2023}}
            \begin{itemize}
                \item На основе данных из рекламного кабинета создал отчет по ставкам и ключевым запросам; это помогло эффективно оптимизировать рекламу и повысить ROMI примерно на \textbf{\textasciitilde 10\%};
                \item Внедрил прогноз окупаемости инвестиций на основе исторических данных по категории с сегментацией на платный и органический трафик, учитывая все возможные места продаж; это помогло привлечь более \textbf{\$25k} инвестиций.
            \end{itemize}
        \end{cvsubsection}
    \end{cvsection}
    
    \begin{cvsection}{\Large Образование}
        \begin{cvsubsection}{Центральный университет, Математика и компьютерные науки, MSc}{\textit{2025 -- 2028}}
            \begin{itemize}
                \item Анализ данных \textit{(Продуктовая студия, Мат. статистика, Инженерия данных)}.
            \end{itemize}
        \end{cvsubsection}
        \begin{cvsubsection}{Высшая школа экономики, Факультет компьютерных наук, Дополнительное образование}{\textit{2023 -- 2024}}
            \begin{itemize}
                \item Математика для Data Science \textit{(Математический анализ, Теория вероятностей, Линейная алгебра)}.
            \end{itemize}
        \end{cvsubsection}
        \begin{cvsubsection}{Финансовый университет, Факультет технологий и анализа данных, BSc}{\textit{2021 -- 2028}}
            \begin{itemize}
                \item Информационные технологии и Data Science в экономике \textit{(Теория алгоритмов, Криптография, ML)}.
            \end{itemize}
        \end{cvsubsection}
    \end{cvsection}
    
    \begin{cvsection}{\Large Навыки}
	\vspace{-1.5em}
    \begin{cvsubsection}{}{}{}  
        Python \textit{(Pandas, Plotly, NetworkX, Scikit-learn, Scipy.stats)}; SQL \textit{(BigQuery, Greenplum, PostgreSQL)}; Airflow; GitHub; BI \textit{(Tableau, Superset)}; Figma; Jira \textit{(Atlassian)}.
    \end{cvsubsection}
	\end{cvsection}
    
\end{document}









